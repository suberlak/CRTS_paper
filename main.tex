% mnras_template.tex
%
% LaTeX template for creating an MNRAS paper
%
% v3.0 released 14 May 2015
% (version numbers match those of mnras.cls)
%
% Copyright (C) Royal Astronomical Society 2015
% Authors:
% Keith T. Smith (Royal Astronomical Society)

% Change log
%
% v3.0 May 2015
%    Renamed to match the new package name
%    Version number matches mnras.cls
%    A few minor tweaks to wording
% v1.0 September 2013
%    Beta testing only - never publicly released
%    First version: a simple (ish) template for creating an MNRAS paper

%%%%%%%%%%%%%%%%%%%%%%%%%%%%%%%%%%%%%%%%%%%%%%%%%%
% Basic setup. Most papers should leave these options alone.
\documentclass[fleqn,usenatbib]{mnras}  % a4paper,

% MNRAS is set in Times font. If you don't have this installed (most LaTeX
% installations will be fine) or prefer the old Computer Modern fonts, comment
% out the following line
%\usepackage{newtxtext,newtxmath}
%\usepackage{txfonts}
% Depending on your LaTeX fonts installation, you might get better results with one of these:
\usepackage{mathptmx}
%\usepackage{txfonts}


% Use vector fonts, so it zooms properly in on-screen viewing software
% Don't change these lines unless you know what you are doing
\usepackage[T1]{fontenc}
\usepackage{ae,aecompl}
\usepackage{slashbox}

%%%%% AUTHORS - PLACE YOUR OWN PACKAGES HERE %%%%%

% Only include extra packages if you really need them. Common packages are:
\usepackage{graphicx}	% Including figure files
\usepackage{amsmath}	% Advanced maths commands
\usepackage{amssymb}	% Extra maths symbols

%%%%%%%%%%%%%%%%%%%%%%%%%%%%%%%%%%%%%%%%%%%%%%%%%%

%%%%% AUTHORS - PLACE YOUR OWN COMMANDS HERE %%%%%

% Please keep new commands to a minimum, and use \newcommand not \def to avoid
% overwriting existing commands. Example:
%\newcommand{\pcm}{\,cm$^{-2}$}	% per cm-squared

%%%%%%%%%%%%%%%%%%%%%%%%%%%%%%%%%%%%%%%%%%%%%%%%%%

%%%%%%%%%%%%%%%%%%% TITLE PAGE %%%%%%%%%%%%%%%%%%%

% Title of the paper, and the short title which is used in the headers.
% Keep the title short and informative.
\title[Quasar Variability]{Solving the puzzle of discrepant variability on monthly time scales implied by SDSS and CRTS datasets}

% The list of authors, and the short list which is used in the headers.
% If you need two or more lines of authors, add an extra line using \newauthor
\author[K. Suberlak et al.]{
Krzysztof Suberlak,$^{1}$\thanks{E-mail: suberlak@uw.edu}
\v{Z}eljko Ivezi\'c, $^{1}$
Chelsea L. MacLeod,$^{2}$
Matthew Graham,$^{3}$ 
\newauthor
$\, \,  $Branimir Sesar$^{4}$
\\
% List of institutions
$^{1}$Department of Astronomy, University of Washington, Seattle, WA, United States\\
$^{2}$Institute for Astronomy, University of Edinburgh, Royal Observatory, Edinburgh, United Kingdom\\
$^{3}$Center for Data-Driven Discovery, California Institute of Technology, Pasadena, CA, United States\\
$^{4}$National Optical Astronomy Observatory, Tucson, AZ, United States.
}

% These dates will be filled out by the publisher
\date{Accepted XXX. Received YYY; in original form ZZZ}

% Enter the current year, for the copyright statements etc.
\pubyear{2016}

% Don't change these lines
\begin{document}
\label{firstpage}
\pagerange{\pageref{firstpage}--\pageref{lastpage}}
\maketitle

% Abstract of the paper
\begin{abstract}

We present improved error analysis for the 3,800 CRTS (Catalina Real-Time Transient Survey) optical  light curves for quasars from the Sloan Digital Sky Survey Stripe 82 catalog. SDSS imaging survey has provided a time-resolved photometric  dataset which greatly improved our understanding of the quasar optical continuum variability: data for monthly and longer timescales  are consistent with a damped random walk. Recently, newer data  obtained by CRTS provided  puzzling evidence for enhanced variability, compared to SDSS results, on monthly time scales. Quantitatively, SDSS results predict  about $0.06$ mag root-mean-square variability for timescales below 50 days, while CRTS data show about a factor of two larger rms for spectroscopically confirmed SDSS quasars. Our analysis presented here has successfully resolved this discrepancy as due to slightly underestimated photometric uncertainties provided by the CRTS image processing pipelines. The photometric error correction factors, derived from detailed analysis of non-variable SDSS standard stars that were re-observed by CRTS, are about $20-30\%$, and result in a quasar variability behavior implied by the CRTS data that is fully consistent with earlier SDSS results.


\end{abstract}


%%%%%%%%%%%%%%%%%%%%%%%%%%%%%%%%%%%%%%%%%%%%%%%%%%

%%%%%%%%%%%%%%%%% BODY OF PAPER %%%%%%%%%%%%%%%%%%

\section{Introduction}
Variability can be used to both characterize and select quasars  in sky surveys . Ahtough various timescales of variability can be linked to physical parameters, such as accretion disk viscosity, or corona geometry (Kelly+2011, Graham+2014), the physical mechanism remains elusive, and most viable explanations include accretion disk instabilities (Kawaguchi + 1998), surface thermal fluctuations from magnetic field turbulence (Kelly+2009), coronal x-ray heating (Kelly+2011) (see Kozlowski 2016 for review).
The diversity of  physical scenarios explaining  the origin of quasar variability led to a  plethore of ways to characterize the varying brightness. The two most widely used approaches to describe variability of quasars  are damped random walk (DRW) and structure function. The DRW model is more suited for lightcurves with typical cadence of days (Zu+2013, Koz\l{}owski+2016), whereas an ensemble SF analysis is better for sparsely sampled lightcurves (Hawkins 2002, Vanden Berk 2004 , de Vries 2005, Schmidt 2010, Graham 2014, or review in Kozlowski 2016). Although CRTS data has been used for both an SF and DRW analysis [CITE WHERE],  we use the SF approach, as more robust for sparsely sampled lightcurves than the DRW. [ For a recent overview of the context for variability studies, see Lawrence+2016. ]

Although SF can be defined  in a variety of ways, it can be characterized by a simple broken power law (Schmidt+2010). At short timescales,  the variability amplitude increases, following a steeper power law index,  until the power law index starts to flatten  above the characteristic timescale  $\tau$.   This knee in the power law  description may correspond to a transition from the stochastic thermal  process driving  the variability, to the physical response of the disk that successfully  dampens the amplitude on longer timescales (Lawrence+2016,  Kelly+2007,2009,2011, Collier+Peterson 2001 (linking the amplitude to the black hole mass)). Indeed,  in x-rays quasars are described by a broken  power law, where the  break timescale is linked to the size of x-ray emitting region (Kelly+2011 in Graham+2014). 

Altough previous studies found that  $\tau > 100$ days  (MacLeod+2011,  Koz\l{}owski+2016), recently, Graham+2014 used the SF, DRW, and Slepian Wavelet Variance (SWV) analysis for CRTS and SDSS S82 lightcurves. Using SWV on CRTS data they found characteristic time scales at   QSO rest frame of $\log_{2}(\tau) = 5.75$)  54 days,  and ($\log_{2}(\tau)=8.2$) -  294 days. Additionally, using this method on S82 data they found a peak at $\log_{2}(\tau) = 7.25$ , and for OGLE  at $\log_{2}(\tau)$ $> 6$ . The short timescale of   $\tau = 54$ days is surprising, as it is shorter by a factor of two than any previous estimates (MacLeod+2011, Zu+2013). We set out to reanalyze the CRTS data, and investigate the plausibility of this discrepant timescale. 


\section{Data Sets}

We study stars and quasars from  Stripe 82, using the  Sloan Digital Sky Survey   and the Catalina Real-time Transient Survey data. Stripe 82 is a large (over 100 deg$^{2}$), repeatedly observed reqion of the equatorial sky ($22^{h} 24^{m} < \mathrm{RA} < 04^{h} 08^{m}$ and $\mathrm{| Dec |} < 1.27 deg$)    (Sesar+2007, Suveges+2012). 

\subsection{Sloan Digital Sky Survey (SDSS)}
We use the SDSS catalog data with  robust,  five-band near-simultaneous  photometry for  9258  quasars,  and 1006849 standard stars - we use it for photometric color information and sample selection.  The  quasar catalog contains spectroscopically confirmed quasars from the SDSS Data Release 7 (Abazajian+2009), based on SDSS Quasar  Catalog IV (Schneider+2008, VizieR Online Data Catalog, 7252, 0), and was compiled by Macleod+2011 (see \footnote{\url{http://www.astro.washington.edu/users/ivezic/cmacleod/qso_dr7/Southern.html}} ) .
The standard stars catalog ver. 2.6 was compiled by Ivezic+2007 (see \footnote{\url{http://www.astro.washington.edu/users/ivezic/sdss/catalogs/stripe82.html}}) .

\subsection{Catalina Real-time Transient Survey (CRTS)}
The CRTS data consists of white light (no filter) lightcurves  - it was designed to find near-Earth objects, hence short intra-night cadence, to allow a rapid follow up (Graham+2015).  Three survey telescopes (0.7m Catalina Sky Survey Schmidt in Arizona,  1.5m Mount  Lemmon Survey telescope in Arizona, and the 0.5m Siding Spring Survey Schmidt in Australia) were equipped with identical, 4kx4k CCDs (see Djorgovski+2011 for technical details CRTS), taking 4 exposures each night.
Although in principle white light magnitudes can be calibrated to Johnson V-band zero point (Drake+2013), we found it a redundant step for our analysis. 

For our study we used a sample of 7932 spectroscopically confirmed S82  quasars,   prepared by B. Sesar, from  the Data Release 2, based on the list by MacLeod+2011.  The majority (96\%) of  CRTS quasar lightcurves span the time of 7-9 years, with a mean sampling of 1 to 4 times per night,  on 70 nights on average (see Fig.1, upper-left panel ).  Mean interval between epochs is 209 days (Fig.1 bottom-right panel), and the mean quasar magnitude is 19.5 mag. 

Our comparison sample consists of CRTS lightcurves of 52133 randomly chosen  10\% of the S82 standard stars catalog ver.2.6 (Ivezic+2007), extracted by B. Sesar from the CRTS DR2.


\subsection{Catalog Matching}
To enrich the information about each CRTS object with SDSS color information, we matched positionally CRTS data to SDSS data within S82 using astropy \verb|match_to_catalog_sky|  routine\footnote{\url{http://docs.astropy.org/en/stable/coordinates/matchsep.html\#matching-catalogs}}.  
Given that the SDSS catalog has 100 times more stars than CRTS, we found an SDSS counterpart to all  CRTS stars within 0.01 arcsec matching radius. However, since there is less SDSS quasars in the S82 catalog, and we found an SDSS counterpart to 7586 CRTS quasars within 0.36 arcsec. We  ignored  15 quasars for which no SDSS counterpart was found within 1 arcsec.  


\subsection{Preprocessing}

Because we are not investigating hourly timescales, prior to structure function analysis we day-averaged all  CRTS light curves. Similarly to  Charisi+2016, we replaced the magnitudes  from the  $j$-th  night by their mean weighted by the inverse square of error:
\begin{equation}
 m_{j} = \langle m_{ij} \rangle = \frac{\sum_{i=1}^{N} {w_{i,j} m_{i,j}} } {\sum_{i=1}^{N} {w_{i,j}} }
\end{equation}
where $i=1...N$, is the number of observations per night, with weights $w_{i,j} = e_{i,j}^{-2}$ .  
We estimate the error on the weighted mean by the inverse square of the sum of weights:  
\begin{equation}
\sigma_{j}(m_{j}) = \frac{1}{\sqrt{\sum_{i}{w_{i,j}}}}
\end{equation}   [ any CITATION ? is it a standard thing to do ? ] Finally, to avoid unphysically small errors, we added in quadrature 0.01 mag  to $\sigma_{j}$ if $\sigma_{j} < 0.02 \,  mag $. 


\subsection{Selection}
We selected both quasars and stars using a combination of information from SDSS and CRTS. To find magnitude difference between different days we first required that the raw lightcurve has more than 10 epochs, which from initial 52131 stars and 7932 quasars left 49385 stars and 7707 quasars. We thus also removed those lightcurves with less than 10 days of observations, leaving 7601 quasars and 48250 stars.  We also required that the lightcurve-average of day-error ($\langle \sigma_{j}(m_{j}) \rangle$) be less than $0.3^{mag}$. Since the raw distribution of errors   peaks at lower values (mean of 0.19 for stars and 0.22 for quasars) than the distribution of the weighted mean errors (mean of 0.13 for stars and 0.15 for quasars), this cut only removes less than 10\% of lightcurves. Our final sample consists of 7108 quasars and 42864 stars.

\section{Analysis Methods}
To analyze the quasar and stellar lightcurves we consider a relationship between measured data $m_{j}$  at times $t_{j}$, and its "copy" shifted by $\Delta  t$ (Koz\l{}owski+2016). We bin the data in bins of $\Delta t$, and calculate statistics that characterize  the variability of stars and quasars. By splitting our sample of stars by color into "blue" and "red" we compare their variability properties to those of quasars in three magnitude bins.

\subsection{Structure Function}
The structure function is a well-studied approach to characterizing lightcurves (Vanden Berk +2004,  de Vries+ 2005, Koz\l{}owski+2016, Graham+2015) . To avoid the uncertain redshift estimate based on SDSS spectra that would be required to correct to the rest-frame variability, we use the observed frame time lags (like Schmidt+2010)  (see Kozlowski+2016).  The  magnitude difference $\Delta m_{j,k}$ for $\Delta t_{j,k} = |t_{j} - t_{k}|$, and the errors added in quadrature :  $\sigma_{j,k}^{2} = \sigma_{j}^{2} + \sigma_{k}^{2}$. 


We prepared 'master' files for stars and quasars, with $\Delta m_{j,k}$, $\Delta t_{j,k} $, $\sigma_{j,k}$.  After day-averaging,  given that the median lightcurve length is  70  days,  each object on average has $\sum_{i=2}^{70}{(i-1)} = 2415 ~ 2E3$  pairwise magnitude difference points. Therefore selecting to study eg.  1000 Quasars,  we have approximately 2E6  magnitude differences.    We study statistical properties of magnitude differences vs time lag by calculating various properties in linearly spaced  bins of $\Delta t_{j,k} $.  We calculated statistical descriptors for each bin, as described in (SECTION ABOUT STATISTICS).    We found that  binning does not affect significantly the shape of structure function nor our conclusions about error properties, and that  choosing 200 bins was optimal for our purpose. 

We characterized the distribution of $\Delta m_{j,k}$  vs  $\Delta t_{j,k}$ by calculating statistics for each bin:  the rms standard deviation $\sigma_{stdev} = \sqrt(mean(abs(x - x.mean())**2))$, the standard deviation based on the interquartile range: $\sigma_{G} = 0.714 IQR $, where IQR = 25% Z - 75% Z, if Z i the sorted distribution of $\Delta m_{j,k}$,   finally $\sigma$, $\mu$, following   [Ivezic+2013 : AstroML], which represent the dispersion of the underlying distribution, and its mean.  

For   each delta(tau) bin, to calculate  sigma and mu we analysed a  distribution of approximate values of those parameters for  1000 boostrapped resamples of the $(m_{ij}, e_{ij})$ data.  The maximum and minimum  of this distribution served as upper and lower bound on calculation of the full solution.  We find that, as in AstroML Fig. 5.8, approximate calculation slightly underestimates sigma, and thus we chose to use full solution, based on marginalizing the log-likelihood of various values of sigma that could describe the $(m_{ij}, e_{ij})$ ensemble.  [ DESCRIBE IN MORE DETAIL, AS IN PAPER II ?  ] 


%%%%%%%%%%%%%%%%%%%%%%%%%%%%%%%%%%%%%%%%%%%%%%%%%%

%%%%%%%%%%%%%%%%%%%% REFERENCES %%%%%%%%%%%%%%%%%%

% The best way to enter references is to use BibTeX:

%\bibliographystyle{mnras}
%\bibliography{example} % if your bibtex file is called example.bib

%%%%%%%%%%%%%%%%%%%%%%%%%%%%%%%%%%%%%%%%%%%%%%%%%%


% Don't change these lines
\bsp	% typesetting comment
\label{lastpage}
\end{document}

% End of mnras_template.tex
