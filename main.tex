% mnras_template.tex
%
% LaTeX template for creating an MNRAS paper
%
% v3.0 released 14 May 2015
% (version numbers match those of mnras.cls)
%
% Copyright (C) Royal Astronomical Society 2015
% Authors:
% Keith T. Smith (Royal Astronomical Society)

% Change log
%
% v3.0 May 2015
%    Renamed to match the new package name
%    Version number matches mnras.cls
%    A few minor tweaks to wording
% v1.0 September 2013
%    Beta testing only - never publicly released
%    First version: a simple (ish) template for creating an MNRAS paper

%%%%%%%%%%%%%%%%%%%%%%%%%%%%%%%%%%%%%%%%%%%%%%%%%%
% Basic setup. Most papers should leave these options alone.
\documentclass[fleqn,usenatbib]{mnras}  % a4paper,

% MNRAS is set in Times font. If you don't have this installed (most LaTeX
% installations will be fine) or prefer the old Computer Modern fonts, comment
% out the following line
%\usepackage{newtxtext,newtxmath}
%\usepackage{txfonts}
% Depending on your LaTeX fonts installation, you might get better results with one of these:
\usepackage{mathptmx}
%\usepackage{txfonts}


% Use vector fonts, so it zooms properly in on-screen viewing software
% Don't change these lines unless you know what you are doing
\usepackage[T1]{fontenc}
\usepackage{ae,aecompl}
\usepackage{slashbox}

%%%%% AUTHORS - PLACE YOUR OWN PACKAGES HERE %%%%%

% Only include extra packages if you really need them. Common packages are:
\usepackage{graphicx}	% Including figure files
\usepackage{amsmath}	% Advanced maths commands
\usepackage{amssymb}	% Extra maths symbols

%%%%%%%%%%%%%%%%%%%%%%%%%%%%%%%%%%%%%%%%%%%%%%%%%%

%%%%% AUTHORS - PLACE YOUR OWN COMMANDS HERE %%%%%

% Please keep new commands to a minimum, and use \newcommand not \def to avoid
% overwriting existing commands. Example:
%\newcommand{\pcm}{\,cm$^{-2}$}	% per cm-squared

%%%%%%%%%%%%%%%%%%%%%%%%%%%%%%%%%%%%%%%%%%%%%%%%%%

%%%%%%%%%%%%%%%%%%% TITLE PAGE %%%%%%%%%%%%%%%%%%%

% Title of the paper, and the short title which is used in the headers.
% Keep the title short and informative.
\title[Quasar Variability]{Solving the puzzle of discrepant variability on monthly time scales implied by SDSS and CRTS datasets}

% The list of authors, and the short list which is used in the headers.
% If you need two or more lines of authors, add an extra line using \newauthor
\author[K. Suberlak et al.]{
Krzysztof Suberlak,$^{1}$\thanks{E-mail: suberlak@uw.edu}
\v{Z}eljko Ivezi\'c, $^{1}$
Chelsea L. MacLeod,$^{2}$
Matthew Graham,$^{3}$ 
\newauthor
$\, \,  $Branimir Sesar$^{4}$
\\
% List of institutions
$^{1}$Department of Astronomy, University of Washington, Seattle, WA, United States\\
$^{2}$Institute for Astronomy, University of Edinburgh, Royal Observatory, Edinburgh, United Kingdom\\
$^{3}$Center for Data-Driven Discovery, California Institute of Technology, Pasadena, CA, United States\\
$^{4}$National Optical Astronomy Observatory, Tucson, AZ, United States.
}

% These dates will be filled out by the publisher
\date{Accepted XXX. Received YYY; in original form ZZZ}

% Enter the current year, for the copyright statements etc.
\pubyear{2016}

% Don't change these lines
\begin{document}
\label{firstpage}
\pagerange{\pageref{firstpage}--\pageref{lastpage}}
\maketitle

% Abstract of the paper
\begin{abstract}

We present improved error analysis for the 3,800 CRTS (Catalina Real-Time Transient Survey) optical quasar light curves from the Sloan Digital Sky Survey Stripe 82 catalog. SDSS imaging survey has provided a time-resolved photometric  dataset which greatly improved our understanding of the quasar optical continuum variability: data for monthly and longer timescales  are consistent with a damped random walk. Recently, newer data  obtained by CRTS (Catalina Real-Time Transient Survey) provided  puzzling evidence for enhanced variability, compared to SDSS results, on monthly time scales. Quantitatively, SDSS results predict  about $0.06$ mag rms variability for timescales below 50 days, while CRTS data show about a factor of two larger root-mean-square  for spectroscopically confirmed SDSS quasars. Our analysis presented here has successfully resolved this discrepancy as due to slightly underestimated photometric error estimates provided by the CRTS image processing pipelines. The photometric error correction factors, derived from detailed analysis of non-variable SDSS standard stars that were re-observed by CRTS, are about $20-30\%$, and result in a quasar variability behavior implied by the CRTS data fully consistent with earlier SDSS results.


\end{abstract}


%%%%%%%%%%%%%%%%%%%%%%%%%%%%%%%%%%%%%%%%%%%%%%%%%%

%%%%%%%%%%%%%%%%% BODY OF PAPER %%%%%%%%%%%%%%%%%%

\section{Introduction}
Variability can be used to both characterize and select quasars  in sky surveys . Ahtough various timescales of variability can be linked to physical parameters, such as accretion disk viscosity, or corona geometry (Kelly+2011, Graham+2014), the physical mechanism remains elusive, and most viable explanations include accretion disk instabilities (Kawaguchi + 1998), surface thermal fluctuations from magnetic field turbulence (Kelly+2009), coronal x-ray heating (Kelly+2011) (see Kozlowski 2016 for review).
The diversity of  physical scenarios explaining  the origin of quasar variability led to a  plethore of ways to characterize the varying brightness. The two most widely used approaches to describe variability of quasars  are damped random walk (DRW) and structure function. The DRW model is more suited for lightcurves with typical cadence of days (Zu+2013, Koz\l{}owski+2016), whereas an ensemble SF analysis is better for sparsely sampled lightcurves (Hawkins 2002, Vanden Berk 2004 , de Vries 2005, Schmidt 2010, Graham 2014, or review in Kozlowski 2016). Although CRTS data has been used for both an SF and DRW analysis [CITE WHERE],  we use the SF approach, as more robust for sparsely sampled lightcurves than the DRW. [ For a recent overview of the context for variability studies, see Lawrence+2016. ]

Although SF can be defined  in a variety of ways, it can be characterized by a simple broken power law (Schmidt+2010). At short timescales,  the variability amplitude increases, following a steeper power law index,  until the power law index starts to flatten  above the characteristic timescale  $\tau$.   This knee in the power law  description may correspond to a transition from the stochastic thermal  process driving  the variability, to the physical response of the disk that successfully  dampens the amplitude on longer timescales (Lawrence+2016,  Kelly+2007,2009,2011, Collier+Peterson 2001 (linking the amplitude to the black hole mass)). Indeed,  in x-rays quasars are described by a broken  power law, where the  break timescale is linked to the size of x-ray emitting region (Kelly+2011 in Graham+2014). 

Altough previous studies found that  $\tau > 100$ days  (MacLeod+2011,  Koz\l{}owski+2016), recently, Graham+2014 used the SF, DRW, and Slepian Wavelet Variance (SWV) analysis for CRTS and SDSS S82 lightcurves. Using SWV on CRTS data they found characteristic time scales at   QSO rest frame of (log2(tau) = 5.75)  54 days,  and (log2(tau) =8.2)  294 days. Additionally, using this method on S82 data they found a peak at log2(tau) = 7.25 , and for OGLE  at log2(tau) > 6 . The short timescale of   $\tau = 54$ days is surprising, as it is shorter by a factor of two than any previous estimates (MacLeod+2011, Zu+2013). We set out to reanalyze the CRTS data, and investigate the plausibility of this discrepant timescale. 




%%%%%%%%%%%%%%%%%%%%%%%%%%%%%%%%%%%%%%%%%%%%%%%%%%

%%%%%%%%%%%%%%%%%%%% REFERENCES %%%%%%%%%%%%%%%%%%

% The best way to enter references is to use BibTeX:

%\bibliographystyle{mnras}
%\bibliography{example} % if your bibtex file is called example.bib

%%%%%%%%%%%%%%%%%%%%%%%%%%%%%%%%%%%%%%%%%%%%%%%%%%


% Don't change these lines
\bsp	% typesetting comment
\label{lastpage}
\end{document}

% End of mnras_template.tex
